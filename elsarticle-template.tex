\documentclass[review]{elsarticle}

\usepackage[colorlinks]{hyperref}
\usepackage[colorinlistoftodos]{todonotes}
\usepackage{verbatim}
\usepackage[utf8]{inputenc}
\usepackage[T1]{fontenc}
\usepackage{adjustbox}
\usepackage{multirow}
\usepackage{longtable}
\usepackage{booktabs}
\usepackage{lineno,hyperref}
\usepackage{listings}
\modulolinenumbers[5]

\journal{colleagues for review}

%%%%%%%%%%%%%%%%%%%%%%%
%% Elsevier bibliography styles
%%%%%%%%%%%%%%%%%%%%%%%
%% To change the style, put a % in front of the second line of the current style and
%% remove the % from the second line of the style you would like to use.
%%%%%%%%%%%%%%%%%%%%%%%

%% Numbered
%\bibliographystyle{model1-num-names}

%% Numbered without titles
%\bibliographystyle{model1a-num-names}

%% Harvard 
%\bibliographystyle{model2-names.bst}\biboptions{authoryear}

%% Vancouver numbered
%\usepackage{numcompress}\bibliographystyle{model3-num-names}

%% Vancouver name/year
%\usepackage{numcompress}\bibliographystyle{model4-names}\biboptions{authoryear}

%% APA style
\bibliographystyle{model5-names}\biboptions{authoryear}

%% AMA style
%\usepackage{numcompress}\bibliographystyle{model6-num-names}

%% `Elsevier LaTeX' style
%\bibliographystyle{elsarticle-num}
%%%%%%%%%%%%%%%%%%%%%%%

\begin{document}

\begin{frontmatter}

%% Title, authors and addresses

\title{Differential Gahagan biface morphology in the southern Caddo area and central Texas}

%% use the tnoteref command within \title for footnotes;
%% use the tnotetext command for the associated footnote;
%% use the fnref command within \author or \address for footnotes;
%% use the fntext command for the associated footnote;
%% use the corref command within \author for corresponding author footnotes;
%% use the cortext command for the associated footnote;
%% use the ead command for the email address,
%% and the form \ead[url] for the home page:
%%
%% \title{Title\tnoteref{label1}}
%% \tnotetext[label1]{}
%% \author{Name\corref{cor1}\fnref{label2}}
%% \ead{email address}
%% \ead[url]{home page}
%% \fntext[label2]{}
%% \cortext[cor1]{}
%% \address{Address\fnref{label3}}
%% \fntext[label3]{}


%% use optional labels to link authors explicitly to addresses:
%% \author[label1,label2]{<author name>}
%% \address[label1]{<address>}
%% \address[label2]{<address>}
%% Group authors per affiliation:
\author{Robert Z. Selden, Jr.\textsuperscript{a,b,c*} and John E. Dockall\textsuperscript{d,e}}
\address[1]{Heritage Research Center, Stephen F. Austin State University, United States}
\address[2]{Cultural Heritage Department, Jean Monnet University, France}
\address[3]{ORCID ID \href{http://orcid.org/0000-0002-1789-8449}{0000-0002-1789-8449}}
\address[4]{Prewitt and Associates, Inc., United States}
\address[5]{ORCID ID \href{http://orcid.org/0000-0002-0940-7144}{0000-0002-0940-7144}}
\cortext[cor1]{Corresponding author, Robert Z. Selden, Jr. (zselden@sfasu.edu)}

\begin{abstract}
This investigation follows a recent morphological study of the three largest samples of Gahagan bifaces, and seeks to further characterise local morphological variation using 64 intact or reconstructed Gahagan bifaces from contexts where two or more specimens were recovered at the Gahagan Mound, George C. Davis, and Mounds Plantation sites. Gahagan bifaces are thought to have been imported from central Texas hunter-gatherers. Bifaces were scanned and analysed using the tools of geometric morphometrics to identify whether Gahagan biface morphology differs between and among Caddo features. While these results provide a preview of the morphological differences found in Gahagan bifaces recovered from Caddo archaeological features and from central Texas, they should be considered preliminary until additional specimens can be added to further test this assertion.
\end{abstract}

\begin{keyword}
Archaeology \sep American Southeast \sep Caddo \sep lithics \sep bifaces \sep NAGPRA
\end{keyword}

\end{frontmatter}

\linenumbers

\section*{Gahagan bifaces}

\begin{quote}
The mathematical definition of a ``form'' has a quality of precision which was quite lacking in our earlier stage of mere description; it is expressed in few words, or in still briefer symbols, and these words or symbols are so pregnant with meaning that thought itself is economised \citep[720-721]{RN11532}.    
\end{quote}

This contribution follows a recent study of Gahagan biface morphology that enlisted the three largest samples from the Gahagan Mound (16RR1), George C. Davis (41CE19), and Mounds Plantation (16CD12) sites in the southern Caddo area (Figure 1) \citep{RN11783}. The results of that study indicated a significant difference in shape for Gahagan bifaces found at Mounds Plantation site when compared with those found at the Gahagan Mound and George C. Davis sites \citep[Figure 7]{RN11783}. The test for morphological disparity indicated that the sample from Gahagan Mound occupied a significantly greater range of morphospace than the sample from Mounds Plantation, providing limited evidence for discussions of specialisation and diversity. Morphological integration was also significant, meaning that those traits used to characterise Gahagan biface shape (blade and base) were found to vary in a coordinated manner. The results confirmed the supposition advanced by \citep{RN3684} that the assemblage of Gahagan bifaces from the George C. Davis site compares favourably with those reported from the Gahagan Mound site \citep{RN5274,RN2740}.

\bibliography{mybibfile}

\end{document}